\section{Introduction}

In Regev's celebrated reduction of worst-case lattice problems to
LWE~\cite{DBLP:journals/jacm/Regev09}, the (lack of) tightness
stemming from Lemma 3.6 (Verifying solutions
of LWE), has recently been claimed by Sarkar and Singha
in~\cite{cryptoeprint:2019:728} to be the cause of most of the loss of
tightness for dimensions from $2$ to $187849$, where it is computed to
be so overwhelmingly large that Regev's reduction is completely vacuous for all
parameters in that range. 

\subsection{Previous Work on Tightness}
\label{sec:prev-work}
A previous work by Chatterjee et
al~\cite{DBLP:conf/mycrypt/ChatterjeeKMS16} showed that, for
modulus $q=n^2$, and noise parameter $\alpha=1/(\sqrt{n}\log^{2}(n))$
(as set by Regev in his proposed decision $\lwe$-based public key cryptosystem),
Regev's reduction has, for an algorithm given $m=n^{c}$ samples, that
solves decision $\lwe$ for a fraction $1/n^{d_1}$ of all $\vecs \in
\Z_q^{n}$ with advantage at least $1/n^{d_2}$, has an enormous
tightness gap in solving $\sivp$, namely 
\[O(n^{11+c+d_1+2d_2})\]

While this gives a tightness gap far larger than the cost of the
best-known algorithms in solving $\sivp$ for commonly used
parameters\jnote{TODO: cite},  


%%% Local Variables: 
%%% mode: latex
%%% TeX-master: "regevreduction"
%%% End: 